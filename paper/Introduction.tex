\section{Introduction} \label{sec:bound-work}

\hspace{13pt} In her 1999 book ``Moving Beyond Gender: Intersectionality and Scientific Knowledge,'' Patricia Hill Collins characterizes ``positivist science,'' a paradigm under which scientific knowledge is produced that articulates specific research practices that lead to unbiased and objective truth. ``Claiming the existence of absolute truths and an objective reality structured by invariant rules, positivist science argues that the underlying structure of social as well as physical phenomena can be uncovered.'' However, a well-established and growing body of feminist and intersectional work has crticized these methodologies for ``using quantitative data in simplistic and superficial ways [and] improperly interpreting and overgeneralizing scientific findings'' \cite{collins1999moving}. In this paper, I will characterize how this simplistic and superficial use of data has persisted in the positivist scientific endeavor by focusing in on the process of constructing datasets. Rather than viewing datasets as unargumentative, sterile, or ``raw,'' I argue that datasets, and operations carried out on them, express the social conceptualizations held by their creators. Especially in reference to social categorizations and identities, naming columns; writing descriptions of these columns in codebooks; constructing categories within which subjects must identify (or be assigned); binning entries on a continuous scale into discrete categories; naming these new categories; constructing and naming new categories within columns from pre-existing categories; constructing and naming new columns from pre-existing columns; and assigning numerical (and by extension, ordinal) values to categories is a process of argumentation, claims-making, and boundary work. More specifically, in this paper, I examine how sex, gender, race, and ethnicity are named and encoded as variables in data, and how this process is patterned by the identities held by the datasets' creators.

\section{Literature Review} \label{sec:lit-review}

\subsection{Sex and Gender} \label{sec:sex-gender}

\hspace{13pt} The distinction between sex and gender, acknowledging the presence of feminized and masculinized socialization accompanying a more purely biological ``sex,'' arose from feminist theory in the 1970s. In reference to Gayle Rubin, Richardson writes ``Rubin distinguished between the biological category of `sex' (typically, male or female) and the social roles and expectations of `gender' (such as heterosexual masculinity and femininity). The sex/gender distinction analytically separates the anatomy and physiology of males and females (sex) from the behavioral and cultural expectations associated with the ideals of masculinity and femininity (gender)'' \cite{richardson2013sex}. This definition is not historical, too. As Springer et al. write in 2012, ``The IOM, drawing especially on definitions advanced by the World Health Organization (WHO) and the style manual of the Journal of the American Medical Association (JAMA), defines sex as `The classification of living things, generally as male or female according to their reproductive organs and functions assigned by chromosomal complement' and gender as `A person’s self-representation as male or female, or how that person is responded to by social institutions based on the individual’s gender presentation.'\thinspace'' Rather than encourage the reconciliation of the ways that these two interact, though, this distinction has cemented `sex' within the jurisdiction of biologists, and isolated the biological and social into easily separable categories in which outward-flowing causality is granted only to sex. As the same authors write, ``These definitions lend a superficial sense that sex and gender are distinct domains, even as they give causal and temporal priority to biology (`gender is rooted in biology' but sex is presumably pristine and emerges regardless of environment and experience)'' \cite{springer2012beyond}. While those utilizing gender and gendered socialization as variables in research must reconcile the effects of sex on gender, natural scientists are free to regard sex as a pure, self-evident, and unalterable truth. Richardson reflects on this change similarly, writing ``The sex/gender distinction served to harden the notion of X and Y as `sex itself'... The X and Y came to represent the necessary alter ego of gender fluidity, signifying what nature intended the sexual fate of the infant to be'' \cite{richardson2013sex}.

This is not to say, though, that considerable scholarship has not been devoted to contesting this supposed purity of sex as a biological category. As early as 1977, when feminist biologists Ethel Tobach and Betty Rosoff organized the inaugural conference ``Genes and Gender'' to take on recent pop science arguing the inevitability of `sexed' social roles, scientists have critiqued the usefulness of this distinction \cite{tobach1994challenging}. Further, many scientists have carried out work within the confines of the positivist framework showing the biological effects of gendered socialization on physiology. In 2005, Fausto-Sterling showed these effects in regard to bone development \cite{fausto2005bare}; Jordan-Young and Rumiati made a similar argument about the brain in 2012 \cite{jordan2012hardwired}; many studies in neuroendicrinology demonstrate the biological effects of social status and identity \cite{booth2006testosterone, haneishi2007cortisol, van2006social}. Richardson writes that studies like these show that ``gendered life experiences have material effects on the body. These effects show up, in turn, as biologically based `sex differences'\thinspace'' \cite{richardson2013sex}. In response to these critiques, scholars suggest that scientists ``conceptualize sex/gender as a domain of complex phenomena that are simultaneously biological and social, rather than a domain in which the social and biological `overlap'\thinspace'' \cite{springer2012beyond}.

Despite these critiques, though, many scientists continue to capitalize on this distinction, and subsequent prioritization of sex as a causal mechanism for differences, in research. In one example, four scientists presented research on brain research in a 2014 panel at Barnard College. In a talk early on in the session, Rae Silver says ``when I talk about sex differences... I'm thinking of sex as a biological construct, and I'm thinking of the genetic, hormonal, and metabolic factors. I'm not at all thinking about gender role, or what society tells us is male and female typical, and I'm not at all thinking about gender identity---how we express our experiences of our sexuality---because none of this research speaks to that'' \cite{barnard}. Not only does Silver argue that sex effects can be isolated in her research, but that gender identity is literally defined as the expression of sex differences. I do not cite this example for its strikingness or severity, but precisely the opposite; the way that Silver presents this research is archetypal of the positivist scientific endeavor's treatment of sex/gender. Indeed, in 2018, Hanvivsky et al. write that, in funding and publishing and funding guidelines for epidemiology research, ``criteria fail to recognize the complexity of sex/gender, including the intersection of sex/gender with other key factors that shape health.'' Sex/gender are only sometimes mentioned, and when they are, ``there is wide variation in how sex/gender are conceptualized and how researchers are asked to address the inclusion/exclusion of sex/gender in research.'' Principally, these funding agencies often emphasize representation in those carrying out scientific research while leaving the methodology used to conceptualize sex/gender effects in resesarch results unscrutinized. ``[R]equirements that have been institutionalized within funding agencies tend to prioritize greater male/female equality in research teams and funding outcomes over considerations of sex/gender in research content and knowledge production'' \cite{hanvivsky2018}. As has long been argued in feminist scholarship, representation of those holding marginalized identities in science research is absolutely an important endeavor. This need not come at the cost of criticality about the ways that these social identities are conceptualized in resulting knowledge production, though.

\subsection{Race and Ethnicity} \label{sec:race-ethnic}

\hspace{13pt} The analogous story for race/ethnicity is similarly just as much one of semantic inconsistency as it is of claims-making of biological jurisdiction. Initially, the instability across time and place of racial categories make the uselessness of race as a biological mechanism immediate. ``Since its invention to manage the expansion of European enslavement and the colonization of other peoples, the definitions, criteria, and boundary lines that determine racial categories have constantly shifted over the course of U.S. history'' \cite{roberts2011fatal}. However, the biologization of race as a system of power ``has been part and parcel of racism'' \cite{morning2011nature}. This is not to say, though, that race and racism are at all not real in the social sense; ``[w]hile race is not imaginary---it is a very real way our society categorizes people---its intrinsic origin in biology is. Race is not an illusion. Rather, the belief in intrinsic racial difference is'' \cite{roberts2011fatal}. The effects of these systems of oppression on physiology, too, are entirely real. Rather than emanating from biological underpinnings, ``race stands as a proxy for sociocultural, economic, and particular historical processes and experiences... while the
experience of a racialized life may affect health outcomes, the concept of race itself has no biological or genetic basis'' \cite{lee2009race}.

As we see is often the case with sex/gender, ethnicity is sometimes articulated to be the cultural counterpart of race, and just as frequently sloppily interchanged in attempts at political correctness. A 2001 introductory sociology textbook describes the relationship as such:  ``One involves traits that are biological; the other, cultural... People can fairly easily modify their ethnicity... Assuming people mate with others like themselves, however, racial distinctiveness persists over generations'' \cite{macionis2001sociology}. As is evident in the above quote, this social-biological boundary work leads to a seemingly inevitably biologized definition of race. As Morning writes, in an analysis of various college-level textbooks, the ``sociology textbooks [in her sample] suggest to students that race is a reflection, albeit unfaithful, of real underlying physical difference.'' Even with this distinction seemingly established, though, the use of these terms in practice employs this boundary with much less clarity; ``the term ethnicity is frequently used, even when the groups in question are labelled with traditionally racial identifiers.'' Further, she writes that these textbooks ``seem to have simply borrowed the term ethnicity to replace the word race... In other words, the concepts of race and ethnicity are interchangeable.'' The employment of these concepts in biology textbooks is no more thoughtful, either; ``biology textbooks present definitions of race that are decidedly essentialist... the biology texts ground difference firmly at the genetic level; neither human perception nor social processes play a role'' \cite{morning2011nature}.

This is not to say that this sloppiness is specific to textbooks, or further that the biologization of race is at all antiquated in scientific research. In interviews with many university professors, Morning found ``the most frequent definition of race among biologists was one that treated race as a biological characteristic.'' Still, though, the great variability in the these responses ``strongly refute[s] the claim that scientists have arrived at a consensus about the nature of race'' \cite{morning2011nature}. The pursuit of personalized medicine, too, has revitalized the biological race concept as an ideologically-neutral necessity for the betterment of medical treatment outcomes. As Roberts writes, though, ``[p]redicting drug response based on a patient's race rather than on genetic traits, says Lawrence LESCO of the FDA's Center for Drug Evaluation Research, is `like telling time with a sundial instead of looking at a Rolex watch'\thinspace'' \cite{roberts2011fatal}. The endeavor to identify subpopulations with greater likelihood of responding positively to specific medical treatments is by all means a noble cause---supposing that clinically relevant differences will fall neatly along racial (read: social) divisions, though, creates a proxy under which modern racial psuedoscience can be perpetuated without critique.

Altogether, then, race/ethnicity is ill-defined both as a categorization system and, by extension, a biological mechanism. At the same time, the staying power of biologized race, especially among scientists, remains striking.


\subsection{Hypotheses}\label{sec:hyps}

As a result of the discussion above, describing the boundary work that scientists partake in to delineate between social and biological phenomena, I propose the first hypothesis:

\textit{Hypothesis 1.} Of data purporting to measure sex/gender or race/ethnicity effects, data for use in biological contexts will be more likely to name such columns ``Sex'' or ``Race'' rather than ``Gender'' or ``Ethnicity,'' respectively, than data collected for other purposes.

However, as argued above, despite partaking in this boundary work, positivist science often fails to integrate this social-biological delineation into measurements, data collection, and arguments for causal mechanisms. This leads to the next hypothesis:

\textit{Hypothesis 2.} The distribution of entries in columns measuring sex/gender or race/ethnicity effects will be the same, regardless of which term is used to describe the column.

The operationalization of these hypotheses is described in the next section.


