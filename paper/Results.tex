\section{Results}\label{sec:results}

\hspace{13pt} Beginning with Hypothesis 1, I find that both Hypotheses are directionally supported, but to varying extents. In regard to Hypothesis $1(a)$, I find that datasets from packages inferred to be intended for biological research are $25.8\%$ more likely to refer to their columns measuring sex/gender effects as ``Sex'' rather than ``Gender'' than datasets from packages inferred to be intended for other purposes. This difference is statistically significant ($p < 0.001$). As for Hypothesis $1(b)$, datasets from packages inferred to be intended for biological research are 3.5\% more likely to refer to their columns measuring race/ethnicity effects as ``Race'' rather than ``Ethnicity'' than datasets from packages inferred to be intended for other purposes, a difference which is not statistically significant ($p = 0.110$). While I do not find statistically significant evidence for Hypothesis $1(b)$, I argue that not only is the finding in regard to Hypothesis $1(a)$ statistically significant, but is also practically significant; a difference of $25.8\%$ represents substantial evidence of the boundary work of claims-making of ``purely'' biological phenomena carried out by biologists in construction of datasets.

This difference, of course, could be a practical consequence of biologists actually measuring different quantities. Thus, moving on to Hypothesis 2, I test the difference in distributions of the most common entries in sex/gender and race/ethnicity columns. Starting with Hypothesis $2(a)$ I find that the distribution of entries in columns named ``Sex'' are statistically significantly different from those named ``Race'' ($p = .007$, $\hat{\chi^2} = 10.0$, $df = 2$). However, I argue that this difference is not practically significant. As shown in Figure \ref{fig:common_entries}, ``Female'' and ``Male'' are the most common entries, by far, in columns labeled either ``Sex'' or ``Gender.'' While the proportion of entries labeled ``Male'' is nearly identical in both columns, the main difference in distributions increasing the magnitude of the test statistic is the lesser representation of females in columns labeled ``Sex.'' The fact, though, that terms traditionally describing gender identities like ``Man,'' ``Woman,'' or ``Nonbinary''  make up almost none of the entries in columns labelled ``Gender,'' (2, 2, and 1 of 208 entries in the sample, respectively,) suggests that sex and gender are regarded as interchangeable terms for otherwise identical phenomena in constructing data across the entire sample. In regard to Hypothesis $2(b)$, as Figure \ref{fig:common_entries} reflects,  the text mining procedure used in this study did not adequately capture any shared common entries in columns labeled as ``Race'' or ``Ethnicity.'' As a result, the resulting counts (shown in Table \ref{tab:n_entries_race_ethnicity}) do not satisfy the assumptions of the $\chi^2$ goodness-of-fit test, and it is thus inappropriate to carry out significance tests on the observed data.

